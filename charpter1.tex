\chapter{入门}
\setlength{\parindent}{2em}
\small

依循惯例,我们的第一个 lua 程序也仅仅只打印一句 ``\verb|Hello World|'' :

\begin{verbatim}
        print("Hello World")
\end{verbatim}

如果使用独立的Lua解释器,需要调用解释器(通常叫做 \verb|lua| 或者 \verb|lua5.2|)运行第一个程序,在解释器后面带上包含程序的文本文件名。如果上面的程序保存在 \verb|hello.lua| 文件中,使用如下命令运行:

\begin{verbatim}
        % lua hello.lau
\end{verbatim}

再来一个复杂点的例子,下面的程序定义了一个计算给定数字的阶乘的函数,请求用户输入数字,并打印它的阶乘:

\begin{verbatim}
        -- 定义一个阶乘函数
        function fact (n)
          if n == 0 then
            return 1
          else
            return n * fact(n-1)
          end
        end

        print("enter a number:)
        a = io.read("*n")               -- 读取一个数字
        print(fact(a))
\end{verbatim}

\section{程序块(chunk)}

Lua 执行的代码片段,例如一个文件或者交互模式下的一行代码,都被称为``程序块''。程序块就是命令序列(或者语句)。

Lua 的语句间不需要分隔符,如果你愿意,可以使用分号来分隔语句。个人建议仅在两条或多条语句写在同一行时,使用分号。在 Lua 语法中,换行无意义;如下四个程序块都成立且意义相同:

\begin{verbatim}
        a = 1
        b = a*2

        a = 1;
        b = a*2;

        a = 1; b = a*2

        a = 1 b = a*2                   -- 我很丑,可是我好用☺
\end{verbatim}

程序块可以简单到只有一条语句,就象 ``\verb|Hello World|'' 那个例子。也可以由句子和函数定义(后面我们会看到,它实际上就是赋值操作)混合组成,就象阶乘那个例子。如果你想,程序块可以很大。因为 Lua 也用作数据描述语言,带有几兆字节的程序块并不少见。Lua 解释器在处理大程序块方面完全没有问题。

除了把程序写进文件,还可以在解释器中以交互模式运行。如果不带参数运行 \verb|lua|,将显示交互提示:

\begin{verbatim}
        % lua  
        Lua 5.2 Copyright (C) 1994-2012 lua.org, PUC-Rio
        >
\end{verbatim}

此后输入的每个命令(比如 \verb|print"Hello World"|)都将立即执行。要退出交互模式和解释器,只需输入文件结束控制符(在 \verb|UNIX| 中是 \verb|ctrl-D|,在 \verb|Windows| 中则是 \verb|ctrl-Z|),或者调用操作系统库的 \verb|exit| 函数,即输入 \verb|os.exit()|。

在交互模式下, Lua 通常把输入的每一行视为一个完整的程序块。但是,如果发现某一行不足以构成完整的程序块,它会等待更多的输入,直到形成一个完整的程序块。通过这种方式 ,可以输入类似前面的阶乘函数那样的多行定义。不过,把定义写入文件并调用 Lua 运行这些文件,通常更实用方便。

可以使用 \verb|-i| 参数通知 Lua 在运行给定的程序块后进入交互会话:

\begin{verbatim}
        % lua -i prog
\end{verbatim}

这条命令将运行 \verb|prog| 文件中的程序块,然后显示交互提示。这种方式对于调试和手工测试尤其有用。在本章的最后,将看到独立解释器的其它参数。

运行程序块的另一种方法是使用实时执行文件的 \verb|dofile| 函数。举个例子,假设有包含如下代码的 \verb|lib1.lua| 文件:

\begin{verbatim}
        function norm (x, y)
          return (x^2 + y^2)^0.5
        end

        function twice (x)
          return 2*x
        end
\end{verbatim}

在交互模式下,输入:

\begin{verbatim}
        > dofile("lib1.lua")            -- 载入库文件
        > n = norm(3.4, 1.0)
        > print(twice(n))               --> 7.0880180586677
\end{verbatim}

\verb|dofile| 函数在测试代码片段时也很有用。可以在两个窗口中工作:一个窗口是代码编辑器(比如正编辑 \verb|prog.lua|),另一个则是以交互模式运行 Lua 的控制台。在保存了对代码的修改后,可以在 Lua console 运行 \verb|dofile("prog.lua")| 来加载新代码;然后试验新代码,调用函数并打印结果。

\section{词法规约}

Lua 中的标识符(或名字)可以是任何由字母、数字和下划线构成,但不以数字开头的字符串;例如:

\begin{verbatim}
        i       j       i10     _ij
        aSomewhatLongName       _INPUT
\end{verbatim}

要避免使用以一个下划线开头、带有一个或多个大写字母的标识符(比如 \verb|_VERSION|);他们在 Lua 中被保留作特殊用途。通常,我保留带有一个下划线的标识符用于虚变量。

在旧版本的 Lua 中,字符的概念依赖于本地化环境。但是程序中的一些字符不能在不支持这些本地化的系统中运行。因此,Lua 5.2 仅允许使用 \verb|A-Z| 和 \verb|a-z| 作为标识符可用的字母。

下面这些关键词是保留的;不能把它们用作标识符:

\begin{verbatim}
        and     break   do      else    elseif
        end     false   goto    for     function
        if      in      local   nil     not
        or      repeat  return  then    true
        until   while
\end{verbatim}

Lua 区分大小写: {\bfseries \verb|and|} 是一个保留字,但是 \verb|And| 和 \verb|AND| 是另两个不同的标识符。

注解可以在任意位置以两个连字符开始直到行末。Lua 也支持块注释,以 \verb|--[[|开始,以 \verb|]]|结束\footnote{2.4节将看到,块注释可以比这复杂的多}。一个注释掉代码片段的小窍门就是把代码封在 \verb|[[| 和 \verb|]]| 之间,就像这样:

\begin{verbatim}
        --[[
        print(10)               -- 不起作用(注释掉了)
        --]]
\end{verbatim}

要激活这段代码,只需要在第一行前面添加一个连字符:

\begin{verbatim}
        ---[[
        print(10)               --> 10
        --]]
\end{verbatim}

前一个例子,第一行的 \verb|--[[| 是块注释的起始,而最后一行的两个连字符是注释的一部分。后一个例子,\verb|---[[| 序列开始了一个普通的单行注释,于是 第一行和最后一行变量了独立的注释。在这种情况下,\verb|print| 位于注释外面。

\section{全局变量}

全局变量不需要声明,可以直接使用。访问未初始化的变量不会出错,只会得到一个 \verb|nil| 值:

\begin{verbatim}
        print(b)                --> nil
        b = 10
        print(b)                --> 10
\end{verbatim}

如果给全局变量赋予 \verb|nil| 值,Lua 会假定变量不再被使用:

\begin{verbatim}
        b = nil
        print(b)                --> nil
\end{verbatim}

在这条赋值语句之后,Lua 将回收该变量的内存。

\section{独立解释器}

独立解释器(源文件是 \verb|lua.c|,可执行文件是 \verb|lua|)可直接使用 Lua 的小型程序。这一节介绍它的主要参数。

如果文件第一行以脚本符(`\verb|#|')开头,解释器在载入时将忽略这一行。这个特性支持在类 \verb|UNIX| 系统中以脚本方式运行 Lua 文件。使用下面内容开始脚本(前提是独立解释器程序放在 \verb|/usr/local/bin| 目录中):

\begin{verbatim}
        #!/usr/local/bin/lua
\end{verbatim}

或者使用:

\begin{verbatim}
        #!/usr/bin/env lua
\end{verbatim}

就可以跳过 Lua 解释器直接调用这个脚本文件。

\verb|lua| 命令的用法如下:

\begin{verbatim}
        lua [options] [script [args]]
\end{verbatim}

每个部分都是可选的。前面我们已经见过,不带任何参数调用 \verb|lua| 命令将进入交互模式。 

\verb|-e| 参数允许我们直接在命令行输入代码,就像这样:

\begin{verbatim}
        % lua -e "pring(math.sin(12))"          --> -0.53657291800043
\end{verbatim}

在 \verb|UNIX| 中需要使用双引号防止 shell 解析圆括号。

\verb|-l| 参数加载库。像前面看到的那样,\verb|-i| 在运行其他参数之后进入交互模式。下面的代码先加载 \verb|lib| 库,执行赋值 \verb|x = 10|,最后显示交互提示符:

\begin{verbatim}
        % lua -i -llib -e "x = 10"
\end{verbatim}

在交互模式下,使用等号开头,后面跟着一个表达式,可以打印任何表达式的值:

\begin{verbatim}
        > = math.sin(3)                         --> 0.14112000805987
        > a = 30
        > = a                                   --> 30
\end{verbatim}

此特性可用于将 Lua 当成计算器。

在运行参数前,解释器先寻找名为 \verb|LUA_INIT_5_2| 的系统变量(如果找不到,会寻找 \verb|LUA_INIT|)。如果这些变量包含 \textit{@filename},解释器就运行给定的文件。如果 \verb|LUA_INIT_5_2|(或者 \verb|LUA_INIT|)变量定义不是以 \textit{@} 开头,解释器将其视为 lua 代码运行。 此特性为我们提供了通过这些配置项自定义 Lua 解释器的强大能力。我们可以预载包、改变路径、自定义函数、给函数改名、删除函数,等等。
脚本可以通过预定义的全局变量 \verb|arg| 接收到命令行参数。在调用 \verb|% lua script a b c|时,运行脚本之前,解释器建立一个包含全部命令行参数的表 \verb|arg|。脚本名赋给索引 0,第一个参数(本例中的`\verb|a|')赋给索引 1,其余参数依次赋给对应的索引。显示在脚本前的参数赋给负数索引。例如,思考如下调用:

\begin{verbatim}
        % lua -e "sin=math.sin" script a b
\end{verbatim}

解释器这样整理参数:

\begin{verbatim}
        arg[-3] = "lua"
        arg[-2] = "-e"
        arg[-1] = "sin=math.sin"
        arg[0] = "script"
        arg[1] = "a"
        arg[2] = "b"
\end{verbatim}

多数情况,脚本只使用正数索引(本例中的 \verb|arg[1]| 和 \verb|arg[2]|)。

从 Lua 5.1 开始,脚本也可以通过可变长度参数(vararg)表达式接受接受这些参数。在函数体中,表达式 \verb|...|(三个圆点)代表脚本的参数(第 5.2 节将详细说明可变长度参数表达式。)

\vspace{2em}

\begin{center}
        {\Large 练习}
\end{center}

\vspace{1em}

{\新楷体\textbf{练习1.1:}} 运行阶乘的例子。如果给一个负数会发生什么?修改例子修正这个问题。

{\新楷体\textbf{练习1.2:}} 分别用载入文件和 \verb|-l| 参数并调用 \verb|dofile| 运行第二个例子。你喜欢哪种?

{\新楷体\textbf{练习1.3:}} 你能说出使用 -- 作为注解标识符的其它语言吗?

{\新楷体\textbf{练习1.4:}} 下面字符串中,哪些是有效标识符?

\begin{verbatim}
        ___     _end    End     end     until?  nil     null
\end{verbatim}

{\新楷体\textbf{练习1.5:}} 写一个简单脚本,假定事先不知道的情况下打印脚本名称。