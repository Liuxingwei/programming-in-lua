\chapter{入门}
\setlength{\parindent}{2em}
\small

为了致敬传统,我们的第一个 lua 程序仅仅打印一句 “\verb|Hello World|" :

\begin{verbatim}
        print("Hello World")
\end{verbatim}

如果你使用独立的Lua解释器,你需要调用解释器(通常叫做 \verb|lua| 或者 \verb|lua5.2|)运行你的第一个程序,在解释器后面带上包含你的程序的文本文件名。如果你把上面的程序保存在叫做 \verb|hello| 的文件中,使用如下命令运行它:

\begin{verbatim}
        % lua hello.lau
\end{verbatim}

再来一个更复杂的例子,下面的程序定义了一个计算给定数字的阶乘的函数,请求用户输入数字,并打印它的阶乘:

\begin{verbatim}
        -- 定义一个阶乘函数
        function fact (n)
          if n == 0 then
            return 1
          else
            return n * fact(n-1)
          end
        end

        print("enter a number:)
        a = io.read("*n")               -- 读取一个数字
        print(fact(a))
\end{verbatim}

\section{程序块(chunk)}

Lua 执行的代码片段,例如一个文件或者交互模式下的一行代码,都被称为“程序块”。程序块就是命令序列(或者语句)。

Lua 的语句间不需要分隔符,如果你愿意,可以使用分号来分隔语句。个人建议仅在两条或多条语句写在同一行时,使用分号。在 Lua 语法中,换行无意义;如下四个程序块都成立且意义相同:

\begin{verbatim}
        a = 1
        b = a*2

        a = 1;
        b = a*2;

        a = 1; b = a*2

        a = 1 b = a*2                   -- 我很丑,可是我好用☺
\end{verbatim}

程序块可以简单到只有一条语句,就象 “\verb|Hello World|” 那个例子。也可以由句子和函数定义(后面我们会看到,它实际上就是赋值操作)混合组成,就象阶乘那个例子。如果你想,程序块可以很大。因为 Lua 也用作数据描述语言,带有几兆字节的程序块并不少见。Lua 解释器在处理大程序块方面完全没有问题。

除了把程序写进文件,还可以在解释器中以交互模式运行。如果不带参数运行 \verb|lua|,将得到如下提示:

\begin{verbatim}
        % lua  
        Lua 5.2 Copyright (C) 1994-2012 lua.org, PUC-Rio
        >
\end{verbatim}

此后输入的每个命令(比如 \verb|print"Hello World"|)都将立即执行。要退出交互模式和解释器,只需输入文件结束控制符(在 \verb|UNIX| 中是 \verb|ctrl-D|,在 \verb|Windows| 中则是 \verb|ctrl-Z|),或者调用操作系统库的 \verb|exit| 函数,即输入 \verb|os.exit()|。

在交互模式下, Lua 通常把输入的每一行视为一个完整的程序块。但是,如果发现某一行不足以构成完整的程序块,它会等待更多的输入,直到形成一个完整的程序块。通过这种方式 ,可以输入类似前面的阶乘函数那样的多行定义。不过,把定义写入文件并调用 Lua 运行这些文件,通常更实用方便。

可以使用 \verb|-i| 参数通知 Lua 在运行给定的程序块后进入交互会话:

\begin{verbatim}
        % lua -i prog
\end{verbatim}

这条命令将运行 \verb|prog| 文件中的程序块,然后显示交互提示。这种方式对于调试和手工测试尤其有用。在本章的最后,将看到独立解释器的其它参数。

运行程序块的另一种方法是使用实时执行文件的 \verb|dofile| 函数。举个例子,假设有包含如下代码的 \verb|lib1.lua| 文件:

\begin{verbatim}
        function norm (x, y)
          return (x^2 + y^2)^0.5
        end

        function twice (x)
          return 2*x
        end
\end{verbatim}

在交互模式下,输入:

\begin{verbatim}
        > dofile("lib1.lua")            -- 载入库文件
        > n = norm(3.4, 1.0)
        > print(twice(n))               --> 7.0880180586677
\end{verbatim}

\verb|dofile| 函数在测试代码片段时也很有用。可以在两个窗口中工作:一个窗口是代码编辑器(比如正编辑 \verb|prog.lua|),另一个则是以交互模式运行 Lua 的控制台。在保存了对代码的修改后,可以在 Lua console 运行 \verb|dofile("prog.lua")| 来加载新代码;然后试验新代码,调用函数并打印结果。

\section{词法规约}

Lua 中的标识符(或名字)可以是任何由字母、数字和下划线构成,但不以数字开头的字符串;例如:

\begin{verbatim}
        i       j       i10     _ij
        aSomewhatLongName       _INPUT
\end{verbatim}

要避免使用以一个下划线开头,带有一个或多个大写字母的标识符(比如 \verb|_VERSION|);