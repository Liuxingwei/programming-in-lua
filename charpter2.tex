\chapter{类型和值}

Lua 是动态类型语言。其中没有类型定义,每个值持有自己的类型。

Lua 中有 8 种基本类型: \verb|nil, boolean, number, string, userdata, function, thread, table|。\verb|type| 函数用于获取给定的值的类型名:
\begin{verbatim}
        print(type("Hello world"))              --> string
        print(type(10.4*3))                     --> number
        print(type(print))                      --> function
        print(type(type))                       --> type
        print(type(true))                       --> boolean
        print(type(nil))                        --> nil
        print(type(type(X)))                    --> string
\end{verbatim}
最后一行无论 \verb|X| 的值是什么,结果都是 ``string'',因为 \verb|type| 的值总是字符串。

变量没有预定义类型,每个变量可以包含任意类型的值:
\begin{verbatim}
        print(type(a))                          --> nil (`a' 没有初始化)
        a = 10
        print(type(a))                          --> number
        a = "a string!"
        print(type(a))                          --> string
        a = print                               --> 你没看错,这是正确的!
        print(type(a))                          --> function
\end{verbatim}
注意最后两行:在 Lua 中,函数是一条值;所以我们能够像操作其它值一样操作它。(第六章将看到关于这一特性的更多详情。)

通常,将同一变量用于不同类型,会造成混乱。但是,有时候,审慎地使用这一特性很有用,比如使用 nil 来区别正常的返回值和非正常情况。

