\chapter{类型和值}

Lua 是动态类型语言。其中没有类型定义,每个值持有自己的类型。

Lua 中有 8 种基本类型: \verb|nil, boolean, number, string, userdata, function, thread, table|。\verb|type| 函数用于获取给定的值的类型名:
\begin{verbatim}
        print(type("Hello world"))              --> string
        print(type(10.4*3))                     --> number
        print(type(print))                      --> function
        print(type(type))                       --> type
        print(type(true))                       --> boolean
        print(type(nil))                        --> nil
        print(type(type(X)))                    --> string
\end{verbatim}
最后一行无论 \verb|X| 的值是什么,结果都是 ``string'',因为 \verb|type| 的值总是字符串。

变量没有预定义类型,每个变量可以包含任意类型的值:
\begin{verbatim}
        print(type(a))                          --> nil (`a' 没有初始化)
        a = 10
        print(type(a))                          --> number
        a = "a string!"
        print(type(a))                          --> string
        a = print                               --> 你没看错,这是正确的!
        print(type(a))                          --> function
\end{verbatim}
注意最后两行:在 Lua 中,函数是一等值;所以我们能够像操作其它值一样操作它。(第六章将看到关于这一特性的更多详情。)

通常,将同一变量用于不同类型,会造成混乱。但是,有时候,审慎地使用这一特性很有用,比如使用 nil 来区别正常的返回值和非正常情况。

\section{Nil}

Nil 是一个只带有 \verb|nil| 一个值的类型(单值类型),它的主要作用是区别其他类型。Lua 使用 nil 作为无效值,表示没有有用的有效值。就像我们看到的那样,全局变量被赋值前的默认值就是 nil,还可以用给全局变量赋 nil 值的方式来删除之。

\section{Booleans}

布尔类型有两个值:\verb|false| 和 \verb|true|,用来表示布尔值。但是条件值并未由布尔值独霸:在 Lua 中任何值都能用于条件。条件判断(比如控制结构中的条件)将 \verb|false| 和 \verb|nil| 视为假,其余值均被视为真。尤其要注意的是,在条件判断中Lua 将 0 和空字符串也视为真。

在本书中,我用“假”来表示任务假值,包括布尔值 \verb|false| 和 \verb|nil|。当特指布尔值时,我用“\verb|false|”。同样的规则可以推广到“真”和“\verb|true|”。

\section{Numbers}

数字类型表示实数(双精度浮点数)。Lua 没有整型。

有些人担心使用浮点数,即使简单的处境和比较也可能会产生怪异的结果。事实并非如此。实际上如今所有平台都遵循 IEEE 754 标准中的表示法。遵循这一标准,仅当数字不能精确表示时,才会出现误差。结果不能精确表示时,才会出现舍入。能够精确表示的结果必须给出精确值。

$2^{53}$(约等于$10^{16}$)以内的整数都能用双精度浮点数精确表示。当使用双精度数表示整数时,不会出现舍入错误,除非其绝对值超出了$2^{53}$。Lua 数字能够表示任何 32 位整数而不出现舍入问题。

当然,分数会遇到表示错误。这种情况和使用纸笔没什么区别。如果我们用十进制计算 $1/7$,我们将不得不在某个位置停下。如果我们使用十位数字表示,$1/7$ 将被舍入为 $0.142857142$。如果我们使用十位数字计算 $1/7*7$,结果将是 $0.999999994$,并不是 $1$。此外,在十进制中能够用有穷数字表示的数字,在二进制中将是无穷数。例如,$12.7-20+7.3$用双精度数计算,结果不等于 $0$,因为 $12.7$ 和 $7.3$ 在二进制中都没有精确的有穷表示(详见练习2.3)。

在继续前进之前,切记:整数有精确表示并且没有舍入错误。

多数现代 CPU 浮点计算和整数计算一样快。不过,使用其它的数字类型(比如长整数或单精度数)编译 Lua 也很容易。对像类似嵌入式这样没有浮点数硬件支持的系统非常有用。详情参见发行版中的 \verb|luaconf.h| 文件。

可以书写带有可选小数部分及可选十进制指数部分的数字常量。有效数字示例如下:
\begin{verbatim}
        4       0.4     4.57e-3         0.3e12          5E+20
\end{verbatim}
还可以加上 $0x$ 前缀书写十六进制常量。从 Lua 5.2 开始,十六进制常量也可以有小数和指数(通过 `\verb|p|' 或 `\verb|P|' 前缀),示例如下:
\begin{verbatim}
        0xff (255)      0x1A3 (419)     0x0.2 (0.125)   0x1p-1 (0.5)
        0xa.bp2 (42.75)
\end{verbatim}
(每个常量后面的括号里是它的十进制表示)

\section{Strings}

Lua 中的字符串通常表示一个字符序列。Lua 完全是八位编码,它的字符有包含任意数字编码,包括嵌入的 $0$。这意味着可以将任意二进制数据存入字符串。也可以存储任意表示法(UTF-8, UTF-16等)的 Unicode 字符串。Lua 提供的标准字符串库没有明确支持这些表示法。不过,我们可以放心地处理 UTF-8 字符串,我们会在 21.7 小节详细讨论这个话题。

Lua 中的字符串是不可变的。不能像 C 语言那样修改字符串中的字符;取而代之,可以创建一个字符竽用于编辑,示例如下:

\begin{verbatim}
        a = "one string"
        b = string.gsub(a, "one", "another")            -- 修改部分字符串
        print(a)                                        --> one string
        print(b)                                        --> another string
\end{verbatim}

在 Lua 中,字符串和其它 Lua 对象(table, function 等)一样,属于动态内存管理主题。这表示你无须担心字符串的内存分配和回收,Lua 会帮你做这件事。字符串可以包括单个字符,或者一整本书。对于 Lua,操作 100k 或者 1M 的字符串是很常见的。

可以使用前缀操作符 `\verb|#|'(称为{\新楷体\emph{长度操作符}})计算字符串的长度:
\begin{verbatim}
        a = "hello"
        print(#a)                                       --> 5
        print(#"good\0bye")                             --> 8
\end{verbatim}

{\normalsize{\新楷体{\bfseries{字符串字面量}}}
\vspace{1em}

可以使用单引号或双引号作为字符串字面量的定界符:
\begin{verbatim}
        a = "a line"
        b = 'another line'
\end{verbatim}
它们是等效的。唯一的不同点是在一种引号的内部可以不用转义直接使用另一种引号。

作为一种风格,很多程序员会为同一类字符串使用同一种类型的括号,这里的字符串“种类”与程序相关。例如,操作 XML 的库可能会为 XML 片段保留单引号,因为这些片段常常包含双引号。

Lua 的字符串可以包含如下 C 风格的转义序列:

\begin{verbatim}
        \a      响铃
        \b      退格
        \f      走纸换页
        \n      新行
        \r      回车
        \t      跳格
        \v      垂直跳格
        \\      反斜线
        \"      双引号
        \'      单引号
\end{verbatim}

下面的示例展示了其用法:

\begin{verbatim}
        > print("第一行\n第二行\n\"在双引号里\", '在单引号里'")
        第一行
        第二行
        "在双引号里", '在单引号里'
        
        > print('单引号里的反斜线:\'\\\'')
        单引号里的反斜线:'\'

        > print("相对简单的方法:'\\'")
        相对简单的方法:'\'
\end{verbatim}

我们还能借助转义序列\verb|\ddd| 和 \verb|x\hh|,通用其数字值来指定对应的字符, \verb|ddd| 是三位十进制数字, \verb|hh|是两位十六进制数字。举个复杂点的例子:字面量 \verb|"also\n123\""| 和 \verb|'\97lo\10\04923"'| 拥有相同的值,在 ASCII 系统中:\verb|97| 是 \verb|a| 的 ASCII 编码,\verb|10| 是换行的编码,\verb|49| 则是数字 \verb|1| 的编码(在这个示例中,必须使用三位数字指定 \verb|49|,否则 Lua 就会读成数字 \verb|492|,因为它后面紧接着其它数字。)我们还能把相同的字符串写成 \verb|'\x61\x6c\x6f\x0a\x31\x32\x33\x22'|,通过十六进制编码表示每个字符。

\vspace{1em}
{\normalsize{\新楷体{\bfseries{长字符串}}}
\vspace{1em}

我们还能使用配对的双方括号界定字符串字面量,类似于长注释那样。在这种括号内的字面量可以跨行,并且不会解析转义序列。而且这种形式还会忽略作为首字符的换行。适用于下面这种包含大量代码的字符串:

\begin{verbatim}
        page = [[
        <html>
        <head>
          <title>一个HTML页面</title>
        </head>
        <body>
          <a href="http://www.lua.org">Lua</a>
        </body>
        </html>
        ]]

        write(page)
\end{verbatim}

有时候,可能想要括住类似 \verb|a = b[c[i]]| 这样的代码片段,注意代码中的 \verb|]]|,或者需要括住已经包含长注释的代码。为处理这种情况,可以在两个方括号中间加入任意数量的等号,类似于 \verb|[===[|。这样处理之后,在字符串字面量末尾的关闭方括号中间放入相同数量的等号即可(本例中是 \verb|]===]|)。等号数量不同的括号会被忽略。选择合适数量的等号,就能括住任意字符串而无须在其中加入转义字符。

同样方法也适用于长注释。例如,使用 \verb|--[=[| 开始长注释,它会一直到 \verb|]=]| 才结束。这样更便于注释已经包含长注释的代码片段。

长字符串是代码中包含字符串字面量的理想方式,但是不能用于非文字字面量。虽然 Lua 字符串字面量可以包含任意字符,但在代码中这样做并不是一个好主意:在文本编辑器中可能会遇见问题,类似 ``\verb|\r\n|'' 这样的行结束府,在读取时有可能被处理成 ``\verb|\n|''。更好的替代方案是使用十进制或十六进制的数字转义序列编码二进制数据,例如 ``\verb|x13\x01\xA1\XBB|''。然而,这会给长字符串带来问题:它们会使行变得很长。

对于这种情况, Lua 5.2 提供了转义序列 \verb|\z|:它会跳过其后的空白字符。下面的例子描述了这种用法:
\begin{verbatim}
        data = "\x00\x01\x02\x03\x04\x05\x06\x07\z
                \x08\x09\x0A\x0B\x0C\x0D\x0E\x0F"
\end{verbatim}
第一行末尾的 \verb|\z| 跳过随后的行尾符和下一行的缩进,\verb|\x07| 紧接着 \verb|\x08|。

\section{强制转换}

Lua 提供了数字和字符串之间的运行时自动转换。数值运算符用在字符串上会尝试将其转换为数字:

\begin{verbatim}
        print("10" + 1)                 --> 11
        print("10 + 1")                 --> 10 + 1
        print("-5.3e-10" * "2")         --> -1.06e-09
        print("hello" + 1)              -- 错误(不能转换 "hello")
\end{verbatim}
Lua 仅在算术运行符

相反,如果 Lua 发现一个被期望为字符串的数字,就会将其转换为字符串:

\begin{verbatim}
        print(10 .. 20)                 --> 1020
\end{verbatim}
\verb|..| 是 Lua 中的字符串连接运算符。如果在数字后面写下这个符号,必须用空格隔开,否则 Lua 会认为第一个圆点是小数点。

时至今日,我们仍无法确定在 Lua 中设计这些自动转换是不是一个好主意。作为一条规则,最好不要依赖它们。某些场合,它们会带来便利,但它们也会让语言和程序更加复杂。毕竟,字符串和数字是不同的事物。类似 \verb|10 == "10"| 的比较结果还是假,因为 \verb|10| 是数字,而 ``\verb|10|'' 是字符串。

如果需要将一个字符串明确地转换为数字,可以使用 \verb|tonumber| 函数,如果字符串不能表示一个合适的数字,此函数会返回 \verb|nil|。

\begin{verbatim}
        line = io.read()                -- 读取一行
        n = tonumber(line)              -- 尝试将其转换为数字
        if n == nil then
          error(line .. "不是一个有效数字")
        else
          print(n * 2)
        end
\end{verbatim}

要将数字转换为字符串,可以使用 \verb|tostring| 函数,或者把数字和空字符串连接在一起:
\begin{verbatim}
        print(tostring(10) == "10")     --> true
        print(10 .. "" == "10")         --> true
\end{verbatim}
这种转换总是有效的。

\section{表(Tables)}

table 类型实现为关联数组。关联数组不仅能用数字检索,也能用字符串或语言中的其它任意值(除了 nil)检索。

table 是 Lua 主要的(也是唯一的)数据结构化机制,但其功能很强大。可以用 table 以简单、一致、高效的方式实现普通数组、集合、记录以及其它数据结构。Lua 也用 table 实现包和对象。当写下 \verb|io.read| 时,实际指的是``\verb|io| 模块的 read 函数''。对于 Lua 而言,其含义为:``使用字符串 \verb|read| 作为键检索 \verb|io| 表''。

Lua 中的 table 既不是值也不是变量,而是对象。如果熟悉 Java 或 Scheme 中的数组,就会明白这句话的意思。可以把 table 视为动态分配的对象,程序仅操作其引用(或指针)。Lua 从不在后台私下拷贝或创建新表。不过,你不能在 Lua 中重定义表:没有重定义它的方法。创建新表的方法是借助``\新楷体{\slshape{构造表达式}}'',最简单的格式是写成 {}:

\begin{verbatim}
        a = {}                          -- 创建一个 table 并将其引用赋值给 `a'
        k = "x"
        a[k] = 10                       -- 新建条目,键等于"x",值等于"10"
        a["20"] = "great"               -- 新建条目,键等于"20",值等于"great"
        print(a["x"])                   --> 10
        k = 20
        print(a[k])                     --> "great"
        a["x"] = a["x"] + 1             --> a["x"] 自增
        print(a["x"])                   --> 11
\end{verbatim}

table 总是匿名的。持有 table 的变量和 table 自身之间没有固定关系:

\begin{verbatim}
        a = {}
        a["x"] = 10
        b = a                           -- 'b' 和 'a' 指向同一个 table
        print(b["x"])                   --> 10
        b["x"] = 20
        print(a["x"])                   --> 20
        a = nil                         -- 只剩下 'b' 还指向 table
        b = nil                         -- 没有任何变量指向 table
\end{verbatim}
当程序中没有任何对一个 table 的引用时,Lua 的垃圾回收器将删除 table 并重用其占用的内存。

每个 table 都能使用不同类型的索引存储 值,并且在需要容纳新条目时增大:
\begin{verbatim}
        a = {}                          -- 空 table
        -- 创建 1000 个条目
        for i = 1, 1000 do a[i] = i*2 end
        print(a[9])                     --> 18
        a["x"] = 10
        print(a["x"])                   --> 10
        print(a["y"])                   --> nil
\end{verbatim}
注意最后一行:类似全局变量,未初始化的 table 字段值为 nil。同样类似的地方还有,可以通过给 table 赋值为 nil 来删除它。Lua 使用普通 table 存储全局变量,这并非巧合。第14章将讨论这一主题。

为表示记录,可以使用字段名作为索引。Lua 通过提供 a.name 作为 a["name"] 的语法糖来支持这种表示法。因此,可以用如下更简洁的方式写出上例子的最后几行:

\begin{verbatim}
        a.x = 10                        -- 等同于 a["x"] = 10
        print(a.x)                      -- 等同于 print(a["x"])
        print(a.y)                      -- 等同于 print(a["y"])
\end{verbatim}
对于 Laugh 而言,这两种形式完全相等并且可以随意混用。然而对于人类而言,不同的形式标志了不同的意图。点运算符清晰表明了把 table 用作记录,那是带有固定、预定义地键的集合。字符串符号为 table 可以使用任意字符串作为键提供了方法,或者出于某种理由需要操作特殊的键。

初学者常常会犯把 \verb|a.x| 和 \verb|a[x]| 弄混的错误。第一种形式表示 \verb|a["x"]|,其含义为使用字符串 \verb|``x''| 检索 table。第二种形式是使用变量 \verb|x| 的值检索 table。下面的代码展示了这种区别:
\begin{verbatim}
        a = {}
        x = "y"
        a[x] = 10                       -- 将 10 存储在 "y" 字段中
        print(a[x])     --> 10          -- 获取 "y" 字段的值
        print(a.x)      --> nil         -- 获取 "x" 字段的值(未定义)
        print(a.y)      --> 10          -- 获取 "y" 字段的值
\end{verbatim}

简单地使用带有整数键的表格,即可表示传统数组或列表。不用也不必定义尺寸,只需要定义所需的元素:
\begin{verbatim}
        -- 读取 10 行,存入 table
        a = {}
        for i = 1, 10 do 
                a[i] = io.read()
        end
\end{verbatim}
